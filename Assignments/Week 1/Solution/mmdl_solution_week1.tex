\documentclass[11pt]{article}

% === Page Layout ===
\usepackage[a4paper, margin=1in]{geometry}

% === Packages ===
\usepackage{graphicx}
\usepackage{comment}
\usepackage{fancyhdr}
\usepackage{enumitem}
\usepackage{amssymb}
\usepackage{amsmath}

% === Header/Footer Setup ===
\pagestyle{fancy}
\fancyhf{}
\fancyhead[L]{Multimodal Deep Learning}
\fancyhead[R]{Answer Sheet 1}
\fancyfoot[C]{\thepage}

\begin{document}

% === Title Information ===
\begin{center}
    {\Large \textbf{Multimodal Deep Learning}} \\ [0.3em]
    {\large \textbf{Answer Sheet 1}} \\ [0.3em]
    \large Date: April 29, 2025
\end{center}

\vspace{1cm}

\noindent
\textbf{Exercise 1: Identifying Multi-Modal Pipelines}

\begin{enumerate}[label=(\alph*)]
    \item  {
    \begin{enumerate}[label=\arabic*.]
        \item \textbf{Translation pipeline} because it allows a single-modal input (a photograph of a menu written in Japanese) to be processed and return as an itemized list in English. 
        The input is an image of a menu written in Japanese, while the output is a translated menu list in English.
        \textbf{(Image Captioning / Image-to-Text Generation)}
        \item \textbf{Fusion pipeline} because there are two modalities (an image and a text query) to generate simple classification (Yes/No).
        The reason that this is not an alignment pipeline is that the model needs to be able to reason based on provided inputs.
        \textbf{(Visual Question Answering (VQS) / Visual Reasoning)}
    \end{enumerate}
    }

    \item \textbf{Alignment pipeline} because a text description input can be encoded, while images in databases can be pre-encoded, which both text input and image database can find the similarity in embedding vector.
    This is also called cross-modal retrieval where data is retrieved from different modalities for desired output.
    \textbf{(Cross-Modal Retrieval)}

    \item \textbf{Fusion pipeline} because two input modalities -- a 10-second video clip and a patient’s vital signs table -- should be collated for determining a patient's condition.
    \textbf{(Video and Tabular Fusion)}

    \item \textbf{(Translation pipeline)} because a single modality input is needed to be encoded and decoded to generate an output image.
    \textbf{(Text-to-Image Generation)}
\end{enumerate}

\noindent
\textbf{Exercise 2: Intuition for Manifolds}

\begin{enumerate}
    \item Robot Arm 
    \begin{enumerate}[label=\roman*.]
        \item \textbf{Data Space} \((\mathbf{D})\): \(\mathbb{R}^D \in \{(x_1, y_1, x_2, y_2)\} = \mathbb{R}^4\) with \(x_1\) and \(y_1\) corresponding to robot shoulder and \(x_2\) and \(y_2\) corresponding to elbow bend. 
        \item \textbf{Data Manifold} \((\mathbf{M} \subset \mathbb{R}^D)\): \(\mathbb{R}^M \in \{(x_1, y_1, x_2, y_2) \ | \ \text{robot arm constraints} \}\). 
        \item  \textbf{Intrinsic Dimension (m)}: A two-link robot have at most 2 degree of freedoms (DoFs) -- \(\theta_1 \ \text{and} \ \theta_2\).
        Hence, \(m = 2\).
        \item \textbf{Coordinate Space} \((\mathbf{U} \subset \mathbb{R}^m)\): Hence, \(\mathbb{R}^U \subset \mathbb{R}^m \in \{(\theta_1, \theta_2)\} = \mathbb{R}^2\).        
    \end{enumerate}

    \item MNIST
    \begin{enumerate}[label=\roman*.]
        \item \textbf{Data Space} \((\mathbf{D})\): \(\mathbb{R}^D \in = \mathbb{R}^{1 \times 784}\) because MNIST is a dataset in grayscale and is also flattened. 
        \item \textbf{Data Manifold} \((\mathbf{M} \subset \mathbb{R}^D)\): \(\mathbb{R}^M \in \{(x, y) \ | \ \text{readable pixels (space of valid digit shapes)} \}\). 
        \item  \textbf{Intrinsic Dimension (m)}: By applying Principal Component Analysis (PCA), a flattened image dimension can be reduced. Hence, \(m << 784\) -- depending on data characteristic.
        For simplicity, \(m = 1\) because there are only numbers from 0 to 9 in the MNIST dataset.
        \item \textbf{Coordinate Space} \((\mathbf{U} \subset \mathbb{R}^m)\): Hence, \(\mathbb{R}^U \subset \mathbb{R}^m \in \{i \ | \ i \in 0 \text{ to } 10 \} = \mathbb{R}^1\).        
    \end{enumerate}

    \begin{enumerate}[label=(\alph*)]
        \item Since the intrinsic dimension refers to highly featured vectors, and not all 784 pixels are considered high-feature pixels -- curves, angles, or borderlines between two colors can be considered high-feature, while the black background should not be much emphasized -- so the intrinsic dimension should be much smaller than 784.
    \end{enumerate}
\end{enumerate}

\noindent
\textbf{Exercise 3: Distance in Manifolds}

\begin{enumerate}[label=(\alph*)]
    \item {
        \textbf{Earth.} A = London and B = Sydney.
        \begin{itemize}
            \item \textit{Euclidean Distance} is a straight-line from London to Sydney passing through Earth core.
            \item \textit{Geodesic Distance} is the shortest path along data manifold \textit{M}, which is the path that we expect to see plane flying.
        \end{itemize}
    }
    \item {
        \textbf{Knotted Rope.} Two points on a tangled rope lying on a table.
        \begin{itemize}
            \item \textit{Euclidean Distance} is a straight-line from two points without considering the ball of tangled rope.
            \item \textit{Geodesic Distance} is the shortest path along data manifold \textit{M}, which is the path between two points after untangled.
        \end{itemize}
    }

\end{enumerate}

\noindent
\textbf{Exercise 4: Distance in Manifolds \\}

\noindent
Please see the accompanying file: `exercise1.ipnyb'.

\end{document}